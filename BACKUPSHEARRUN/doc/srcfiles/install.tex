\subsection{Acquiring sf2s}
The program Sturb and related tools are available online through a subversion
repository located at orca.ucsd.edu. It can be checked out remotely using the
command
\vspace{12pt} \begin{center}
svn checkout svn+ssh://USR@orca.ucsd.edu/home/sf2sCodeRepository/sf2s/tags/MRV
\vspace{12pt} \end{center}
Where MRV is the name of the most recent version you want to obtain and USR
should match the user name of a member of the cfdlab group on orca. For a list
of available files use the command
\vspace{12pt} \begin{center}
svn list svn+ssh://USR@orca.ucsd.edu/home/sf2sCodeRepository/sf2s/tags/
\vspace{12pt} \end{center}
You must be a member of this group to be able to commit changes. Anyone can
download the code using the first statement without the USR field. The most
recent stable released version is the largest tagged value of sf2sv*.*. The 
most current stable working version can be found at 
\vspace{12pt}  \begin{center}
orca.ucsd.edu/home/sf2sCodeRepository/sf2s/trunk/
\vspace{12pt} \end{center}

\subsection{Configuring sf2s for your system}
There are a number of sample architecture files in arch/. If the machine you 
want to use is not listed among the ones given, copy the closest one into a new
file and edit it to match your system. FFTW is required and must be linked
using the LIBDIRS, INCDIRS and LIBS flags. The CPPFLAGS allow for
pre-processing to make either a serial or parallel version, as well as the 
creation of separate source directories for the serial code which are free from
pre-processing flags; its value should not be changed. To make the parallel
version, pass the option -DPARALLEL to the PARALLEL flag in the build section.
There are notes at the bottom of the arch files containing helpful information
on building and running the code. Once you have configured an appropriate 
architecture file, copy it from arch/ into arch.in in the main directory.

\subsubsection{Installing FFTW}
Sf2s and related packages use FFTW to perform Fast Fourier Transforms. The 
code uses version 3.* where * is the latest available version. Most if not 
all supercomputers provide FFTW, this section is included for those running
serially or on small clusters. FFTW is available from
http://www.fftw.org/download.html. Installation instructions are available
on that page, they are repeated here for completeness. It is assumed that the
user is working on a Unix based system. Consult the FFTW website if this is
not the case.

\vspace{12pt}
\textbf{Installation Instructions}
\begin{enumerate}
\item Download the latest version from the website: 
  \begin{description}
    \item download fftw-3.*.*.tar.gz
  \end{description}
\item unpack the tar file:
  \begin{description}
    \item tar -xvzf fftw-3.*.*.tar.gz
  \end{description}
\item Configure the installation for your system*: ./configure --prefix=DIR
  \begin{description}
    \item  ./configure --prefix=DIR
  \end{description}
      where DIR is the location you want to install the libraries.
\item Create the make file: make
  \begin{description}
    \item  make
  \end{description}
\item Check that the intallation is set up properly: make check
  \begin{description}
    \item  make check
  \end{description}
\item Install the software: make install
  \begin{description}
    \item  make install
  \end{description}
\end{enumerate}

*The default behavior of ./configure is to install files in /usr/local. It is 
suggested that you put them elsewhere as described above for better
organization and to allow for multiple versions of FFTW to be installed if 
necessary.

\subsubsection{Linking to FFTW}
Once FFTW has been installed, one must update the arch file with the correct
path to its libraries and the relevant fortran file. If FFTW is installed in
/opt/fftw-3.2.1, The two needed files are 
\vspace{12pt} \begin{center}
/opt/fftw-3.2.1/lib/libfftw3.a \\
/opt/fftw-3.2.1/include/fftw3.f
\vspace{12pt} \end{center}
The libfftw3.a file is linked to the main program using the flag LIBDIRS with 
a link to the library. Note that to link to the library one uses -lfftw3 not
-libfftw3!
\vspace{12pt} \begin{center}
LIBDIRS =-L/opt/fftw-3.2.1/lib \\
LIBS    =-lfftw3
\vspace{12pt} \end{center}
The fftw3.f file is linked to the main program using the flag INCDIRS.
\vspace{12pt} \begin{center}
INCDIRS =-I/opt/fftw-3.2.1/include
\vspace{12pt} \end{center}


\subsubsection{Installing ifort}
The Intel Fortran compiler ifort is the compiler of choice for the serial 
version. Note that the serial version will not work with gfortan or g95.

\subsection{Making programs}
There are a number of programs that can be created:
\begin{enumerate}
\item make: makes the DNS solver Sturb.x.
\item make ptest: makes the pressure test program ptest.x
\item make interp1: makes the interpolation program in X1 interpX1.x
\item make interp2: makes the interpolation program in X2 interpX2.x
\item make interp3: makes the interpolation program in X3 interpX3.x
\item make gridp: makes the grid generation program grid.x
\item make mglib: makes the library for the pressure solver libmg.a. This
      should be called after make ptest, otherwise a compilation error will
      occur
\item make clean: cleans the current directory only
\item make cleanall: cleans the current directory and all subdirectories
\end{enumerate}
