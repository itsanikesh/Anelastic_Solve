\documentclass[a4paper,11pt]{article}
%\documentclass[]{article}
% This should be ok, if you get into trouble or something does not work try
% a separate line for each usepackage statement
\usepackage[dvips]{graphicx}
\usepackage[colorlinks=false]{hyperref}
\usepackage{fullpage,epsfig,tabularx,supertabular,amsmath,natbib}
% An error occurs if \usepackage[dvips]{graphicx} is called right before
% this new command statement.
\newcommand{\bmath}{\boldsymbol} 

\title{SF2s User Manual}

\author{Matt de Stadler and Kyle Brucker}

\begin{document}

\maketitle

\tableofcontents

\pagebreak

\section{Introduction}
This document describes the packages available in sf2s. The algorithms
underlying the code are presented as well as a description of the layout of
the code, installation instructions, information on running the code and a 
comprehensive list of subroutines employed is presented.

\section{Numerical Algorithms}
\subsection{DNS Solver}
\subsection{LES Solver}
Coming soon...
\subsection{Multigrid Pressure Solver}
\subsection{Grid Generation Program}
\subsection{Interpolation Program}

\section{Installing the Code}
\subsection{Acquiring sf2s}
The program Sturb and related tools are available online through a subversion
repository located at orca.ucsd.edu. It can be checked out remotely using the
command
\vspace{12pt} \begin{center}
svn checkout svn+ssh://USR@orca.ucsd.edu/home/sf2sCodeRepository/sf2s/tags/MRV
\vspace{12pt} \end{center}
Where MRV is the name of the most recent version you want to obtain and USR
should match the user name of a member of the cfdlab group on orca. For a list
of available files use the command
\vspace{12pt} \begin{center}
svn list svn+ssh://USR@orca.ucsd.edu/home/sf2sCodeRepository/sf2s/tags/
\vspace{12pt} \end{center}
You must be a member of this group to be able to commit changes. Anyone can
download the code using the first statement without the USR field. The most
recent stable released version is the largest tagged value of sf2sv*.*. The 
most current stable working version can be found at 
\vspace{12pt}  \begin{center}
orca.ucsd.edu/home/sf2sCodeRepository/sf2s/trunk/
\vspace{12pt} \end{center}

\subsection{Configuring sf2s for your system}
There are a number of sample architecture files in arch/. If the machine you 
want to use is not listed among the ones given, copy the closest one into a new
file and edit it to match your system. FFTW is required and must be linked
using the LIBDIRS, INCDIRS and LIBS flags. The CPPFLAGS allow for
pre-processing to make either a serial or parallel version, as well as the 
creation of separate source directories for the serial code which are free from
pre-processing flags; its value should not be changed. To make the parallel
version, pass the option -DPARALLEL to the PARALLEL flag in the build section.
There are notes at the bottom of the arch files containing helpful information
on building and running the code. Once you have configured an appropriate 
architecture file, copy it from arch/ into arch.in in the main directory.

\subsubsection{Installing FFTW}
Sf2s and related packages use FFTW to perform Fast Fourier Transforms. The 
code uses version 3.* where * is the latest available version. Most if not 
all supercomputers provide FFTW, this section is included for those running
serially or on small clusters. FFTW is available from
http://www.fftw.org/download.html. Installation instructions are available
on that page, they are repeated here for completeness. It is assumed that the
user is working on a Unix based system. Consult the FFTW website if this is
not the case.

\vspace{12pt}
\textbf{Installation Instructions}
\begin{enumerate}
\item Download the latest version from the website: 
  \begin{description}
    \item download fftw-3.*.*.tar.gz
  \end{description}
\item unpack the tar file:
  \begin{description}
    \item tar -xvzf fftw-3.*.*.tar.gz
  \end{description}
\item Configure the installation for your system*: ./configure --prefix=DIR
  \begin{description}
    \item  ./configure --prefix=DIR
  \end{description}
      where DIR is the location you want to install the libraries.
\item Create the make file: make
  \begin{description}
    \item  make
  \end{description}
\item Check that the intallation is set up properly: make check
  \begin{description}
    \item  make check
  \end{description}
\item Install the software: make install
  \begin{description}
    \item  make install
  \end{description}
\end{enumerate}

*The default behavior of ./configure is to install files in /usr/local. It is 
suggested that you put them elsewhere as described above for better
organization and to allow for multiple versions of FFTW to be installed if 
necessary.

\subsubsection{Linking to FFTW}
Once FFTW has been installed, one must update the arch file with the correct
path to its libraries and the relevant fortran file. If FFTW is installed in
/opt/fftw-3.2.1, The two needed files are 
\vspace{12pt} \begin{center}
/opt/fftw-3.2.1/lib/libfftw3.a \\
/opt/fftw-3.2.1/include/fftw3.f
\vspace{12pt} \end{center}
The libfftw3.a file is linked to the main program using the flag LIBDIRS with 
a link to the library. Note that to link to the library one uses -lfftw3 not
-libfftw3!
\vspace{12pt} \begin{center}
LIBDIRS =-L/opt/fftw-3.2.1/lib \\
LIBS    =-lfftw3
\vspace{12pt} \end{center}
The fftw3.f file is linked to the main program using the flag INCDIRS.
\vspace{12pt} \begin{center}
INCDIRS =-I/opt/fftw-3.2.1/include
\vspace{12pt} \end{center}


\subsubsection{Installing ifort}
The Intel Fortran compiler ifort is the compiler of choice for the serial 
version. Note that the serial version will not work with gfortan or g95.

\subsection{Making programs}
There are a number of programs that can be created:
\begin{enumerate}
\item make: makes the DNS solver Sturb.x.
\item make ptest: makes the pressure test program ptest.x
\item make interp1: makes the interpolation program in X1 interpX1.x
\item make interp2: makes the interpolation program in X2 interpX2.x
\item make interp3: makes the interpolation program in X3 interpX3.x
\item make gridp: makes the grid generation program grid.x
\item make mglib: makes the library for the pressure solver libmg.a. This
      should be called after make ptest, otherwise a compilation error will
      occur
\item make clean: cleans the current directory only
\item make cleanall: cleans the current directory and all subdirectories
\end{enumerate}


\section{Code Layout}
The main directory contains two important files
\begin{enumerate}
\item arch.in: the architecture file which contains configuration settings for 
      all the sf2s packages.
\item Makefile: the Makefile for all the packages of sf2s. This should not be
      modified unless new files are added and you know what you are doing.
\end{enumerate}
The code is broken up into independent directories
\begin{enumerate}
\item arch: contains sample architecture files for computers that have run
      Sturb.
\item decomp: contains files related to creation and usage of an MPI domain
      decomposition.
\item dns: contains files for the DNS solver Sturb.x.
\item doc: contains documentation for the sf2s package.
\item grid: contains files for the grid generation program grid.x.
\item include: contains the module files which define necessary variables for
      the different subprograms.
\item interp: contains files for the interpolation program interpX1.x,
      interpX2.x, interpX3.x.
\item mgMK: contains files for the multigrid pressure solver.
\item mgMK/test: contains files for the pressure solver test program ptest.x.
\item post: contains files for post-processing simulations using Sturb.x.
\item runfiles: contains sample runfiles for the pressure test program and the 
      DNS solver. Parallel and Serial versions are available.
\item scripts: contains useful scripts for ??? KYLE?
\item serialdev: a directory for new serial files. ??? IS THIS NECESSARY KYLE?
\item utils: contains files that are useful for general tasks.

There is an exhaustive list of all subroutines by directory in
Appendix \ref{app:allsubs}.
\end{enumerate}


\section{Running the Code}
\newcommand{\kkzero}{\ensuremath{\frac{k}{k_0}}}
\newcommand{\ddx}[2]{\frac{\partial #1}{\partial #2}}


\subsection{DNS solver}
Running the DNS solver is done with the combination of Sturb.x and Sturb.ini. 
All settings for the simulations are given in Sturb.ini. An example file is
given below:
\input{autogen/sturbini} % path is from main program

\textbf{Restart block options}
\begin{description}
  \item Restart: whether to use a restart file or not
  \item Fname: Restart filename. Only applies if Restart is yes.
  \item iter: Iteration number of the restart file. Only applies if Restart is
 yes.
\end{description}

\textbf{Parameters block options}
\begin{description}
\item Flow: Type of flow. Options are
  \begin{enumerate}
    \item Hshear: Horizontal shear layer
    \item Vshear: Vertical shear layer
    \item Twake: Towed wake
    \item SPwake: Self-propelled wake
    \item PRPwake: Propelled wake
  \end{enumerate}
\item Strat: Density Type.
  \begin{enumerate}
    \item Linear: Linear Stratification
    \item TwoLayer: Uses tanh to create a continuous two layer fluid. Usually
                    used for shear layer simulations.
    \item Jdeep: Jdeep case of ???KYLE OR HIEU WHAT IS THIS?
  \end{enumerate}
\item Reynolds: Reynolds number 
\item Prandtl: Prandtl number.
\item RefDensity: Background density, usually non-dimensionalized to 1.
\item CFL: CFL number for the timestep determination.
\item DeltaT\_max: maximum value for the timestep. Is this used KYLE?
\item Clip: Prevent unnatural values in the density, velocity away from a shear
            layer.
\item IConly: If yes stop the program after the initial conditions are set up
              and the restart file is generated.
\end{description}

\textbf{Gravity block options}
\begin{description}
  \item GT1: Time to set gravity to Ginit.
  \item GT2: Time to set gravity equal to Gfinal. The value is ramped linearly
             from Ginit to GFinal in GT2-GT1. KYLE IS THIS TRUE???
  \item Ginit: Initial value of gravity.
  \item Gfinal: Final version of gravity. Should be entered as a magnitude.
\end{description}

\textbf{Domain block options}
\begin{description}
  \item NX: number of points in the X1 direction.
  \item NY: number of points in the X2 direction.
  \item NZ: number of points in the X3 direction.
\end{description}

\textbf{Decomposition block options}
\begin{description}
  \item X1Procs: number of processors to use in the X1 direction.
  \item X2Procs: number of processors to use in the X2 direction.
  \item X3Procs: number of processors to use in the X3 direction.
\end{description}

\textbf{Iteration block options}
\begin{description}
  \item Start: Number for the first iteration. KYLE what about restarts???
  \item End: Maximum number of iterations.
  \item Restarts: How often to save restart files.
  \item Stats: How often to save the full statistics package.
  \item StatSmall: How often to write the small statistics package.
  \item FirstStat: Whether to write all statistics after the first iteration.
  \item Planes: How often to save planes of data.
  \item Pencils: How often to save pencils of data.
  \item CheckDIV: How often to write the divergence to the screen.
\end{description}

\textbf{Grid block options}
this section will be over-written.
\begin{description}
  \item Lx: Domain Length in the X1 direction.
  \item Ly: Domain Length in the X2 direction.
  \item Lz: Domain Length in the X3 direction.
  \item x0: Where to set x1 = 0. 
  \item y0: Where to set x2 = 0. 
  \item z0: Where to set x3 = 0. 
  \item XGridtype:
  \item YGridtype:
  \item ZGridtype:
\end{description}

\textbf{Output block options}
\begin{description}
  \item RelaxDIR: Directory where pre-relax files will be written out.
  \item ResultDIR: Directory where results will be written out. KYLE is this still needed or should the other files be modified???
  \item TempDIR: Directory where temporary files will be written out. KYLE is this still needed???
  \item PenDIR: Directory where pencil files will be written out.
  \item PlnDIR: Directory where plane files will be written out.
  \item FlowDIR: Directory where restart files will be written out.
  \item StatDIR: Directory where statistics files will be written out.
  \item MGDIR: Directory where grid files will be written out for each
        multigrid level.
  \item GridDIR: Directory where grid files will be written out.
  \item FileExt: The file extension for output. KYLE is this for all files or just some???
  \item WriteGrid: Write the grid to a file.
  \item WriteSponge: Write a file containing the sponge region.
  \item GridWeights: Write out a measure of how stretched the grids are. KYLE???
  \item TKstat: If yes, write statistics in a format to be read by TKstat. This
        is ok for shear layers but not wakes.
  \item StatBIN: If yes, write statistics in binary.
  \item BigIO: If yes, write all output to one large file rather than
        having each processor write its own. Only useful for relatively
        small runs.
\end{description}

\textbf{IO block options}
\begin{description}
  \item IOUT\_MASTER: Where the master should write out messages. 6 is to 
        standard out.
  \item IOUT\_SLAVEX2Procs: Where the slaves should write out messages. 
        0 has them either not write out or write to an error file.
\end{description}

\textbf{Planes block options}
\begin{description}
  \item iPlanes: Values of i where X2-X3 planes should be written out.
  \item jPlanes: Values of j where X1-X3 planes should be written out.
  \item kPlanes: Values of k where X1-X2 planes should be written out.
\end{description}

\textbf{Lines block options}
\begin{description}
  \item iLines: KYLE how are these set up??? Are they working?
  \item jLines: 
  \item kLines: 
\end{description}

\textbf{MGsolver block options}
\begin{description}
  \item nxlevels: Number of multigrid levels in X1.
  \item nylevels: Number of multigrid levels in X2.
  \item nzlevels: Number of multigrid levels in X3.
  \item Smoother: Smoother to use when solving at each multigrid level. Options are 
  \begin{enumerate}
    \item RBPGS: Red-black point Gauss-Seidel iteration. 
    \item LJ: Red-black line Gauss-Seidel iteration. Lines are taken in the X1
          direction. This is well suited to stretched grids. 
  \end{enumerate}
  \item maxcy: Maximum number of V cycles for each call to the pressure solver.
  \item tolmax: Tolerance for the pressure solver.
  \item PreRelax: Number of iterations at each grid level going down the V 
        cycle. Should be entered coarsest to finest, i.e. 12, 10, 8, 4, 4, 3.
  \item PostRelax: Number of iterations at each grid level going up the V 
        cycle. Should be entered coarsest to finest, i.e. 12, 10, 8, 4, 4, 3.
  \item Strict: If yes exchange processor boundary data after each red or 
        black iteration (twice per iteration, once for black, once for red).
  \item Verbose: If yes, output residual at each V cycle. 
\end{description}

\textbf{InitPressure block options}
\begin{description}
  \item Pzero: If yes, set $p=0$ for the initial conditions. KYLE does this work with RESTARTS???
\end{description}

\textbf{ISO block options}
\begin{description}
  \item KO: Scaling parameter for the energy spectrum E(k). KYLE is this right?
  \item K1: KYLE What do these do?
  \item EpsO: 
  \item Spectrum: Which energy spectrum to use. Options are
  \begin{enumerate}
    \item 1: Flat Spectrum $E(k)=1$
    \item 2: $E(k)=\frac{\left( \kkzero \right)^4}  
             {\left(1+\frac{12}{5}\left(\kkzero \right)^2 \right)^\frac{17}{6} }$
    \item 3: $E(k)=\left( \kkzero \right)^4 \exp\left( -2 \left( \kkzero \right) ^2 \right)$
             if $2 \left( \kkzero \right)^2 < 700 $ else $E(k)=0$
    \item 4: $E(k)=\left( \kkzero \right)^2 \exp\left( -2 \kkzero \right)$
             if $2  \kkzero  < 700 $ else $E(k)=0$
    \item 5: $E(k)=\left( \kkzero \right)^\frac{1}{4} \exp\left(-2 \left(\kkzero \right)^2\right)$
             if $2 \left( \kkzero \right)^2 < 700 $ else $E(k)=0$
    \item 6: Kolmogorov Spectrum $E(k)=1.5 \epsilon^ {\frac{2}{3}} k^{\frac{-5}{3}}$
             if $k_0 < k < k_1$ else $E(k)=0$
    \item 7: Pulse Spectrum $E(k)=1$ 
             if $k_0 < k < k_1$ else $E(k)=0$
  \end{enumerate}
\end{description}

\textbf{BroadBand block options}
\begin{description}
  \item Amplify: Magnitude of the initial disturbance. KYLE WHAT ARE THESE FOR SHEAR LAYERS
  \item Radius: Radius of the initial disturbance.
  \item turbICS: If yes, add turbulence to the initial conditions. WHY IS THIS AN OPTION KYLE?
\end{description}

\textbf{Relax block options}
\begin{description}
  \item RelaxTime: Time for pre-relaxing to set up the initial conditions. 
  \item FixMean: Keep mean constant during pre-relaxing. KYLE IS THIS IMPLEMENTED?
\end{description}

\textbf{Mean block options}
\begin{description}
  \item Param1: Amplitude for the wake.
  \item Param2: Radius for the wake.
  \item Param3: Undefined for wake.
  \item Param4: Undefined for wake.
  \item X1center: Location of mean profile center in X1.
  \item X2center: Location of mean profile center in X2.
  \item X3center: Location of mean profile center in X3.
\end{description}

\textbf{Density block options}
\begin{description}
  \item Param1: 
    \begin{description}
      \item Linear case: $\ddx{\rho}{x_3}$
      \item TwoLayer case: $\triangle \rho$
      \item Jdeep case: $J_m$
    \end{description}
  \item Param2: 
    \begin{description}
      \item Linear case: unused
      \item TwoLayer case: $\triangle \omega$
      \item Jdeep case: $J_d$
    \end{description}
  \item Param3: 
    \begin{description}
      \item Linear case: unused
      \item TwoLayer case: unused
      \item Jdeep case: $z_0$
    \end{description}
  \item X3center: Location of mean density center in X3.
\end{description}

\textbf{Sponge block options}
\begin{description}
  \item Ilower: Size of sponge from X1min in grid cells.
  \item Iupper: Size of sponge from X1max in grid cells.
  \item Jlower: Size of sponge from X2min in grid cells.
  \item Jupper: Size of sponge from X2max in grid cells.
  \item Klower: Size of sponge from X3min in grid cells.
  \item Kupper: Size of sponge from X3max in grid cells.
  \item AmpX1: Amplitude of sponge in X1 direction.
  \item AmpX2: Amplitude of sponge in X2 direction.
  \item AmpX3: Amplitude of sponge in X3 direction.
  \item UseSponge: If yes, turn on the sponge.
\end{description}

\textbf{BC block options}
\begin{description}
  \item X1: Set BCs for X1. Options are periodic and read. Periodic makes all
            variables periodic, read lets the BCs for each variable be set 
            below in the appropriate block. If a variable is defined as 
            periodic, the options below are not used.
  \item X2: Set BCs for X2.
  \item X3: Set BCs for X3.
\end{description}

\textbf{X3min block options}
\begin{description}
  \item U1TYPE: Set the type of BC for U1 at X3min. Options are Dirichlet
       (Dirichlet, dirichlet, dir)  and Neumann (Neumann, neumann, neu).
  \item U1VALUE: Assign value for U1 as defined by the BC type above.
  \item FLUCTYPE: Set BCs for the fluctuations. This is important for the 
        statistics. KYLE WHY DO WE HAVE TO DO THIS AGAIN?
  \item FLUCVALUE: Assign value for the fluctuations as defined by the BC type
                   above. 
\end{description}
The other boundary condition variables are defined analogously. 

\subsection{LES solver}
coming soon...

\subsection{Pressure test program}
Running the Pressure solver test program is done with the combination of
ptest.x and ptest.ini. This program can be used to find an optimal domain
decomposition and to tune multigrid settings for the problem to be be studied.
All settings for the simulations are given in ptest.ini. An example file is
given below:
\input{autogen/ptestini}

The ptest.ini file is essentially a smaller version of Sturb.ini. The one major
difference in the two files is the TestCase block in ptest.ini. 
\vspace{12pt}

\textbf{TestCase block options}
\begin{description}
  \item testcase: There are 6 built in test cases with analytic solutions. They are: 
  \begin{enumerate}
    \item dirichletBCs: 1D test with 2 Dirichlet boundary conditions.
      \begin{align*}
        \nabla^2 p = rhs \hspace{1.0in}
         & p(-L/2) = a \\
         & p(L/2) = b \\
         & \text{periodic BCs in all other directions}
      \end{align*}
    \item linearpgrad: 1D test with 2 Dirichlet boundary conditions.
      \begin{align*}
        \nabla^2 p = 0 \hspace{1.0in}
         & p(-L/2) = a \\
         & p(L/2) = b \\
         & \text{periodic BCs in all other directions}
      \end{align*}
    \item 1dirichlet1NeumannBC: 1D test with 1 Dirichlet boundary condition
          and one Neumann boundary condition.
      \begin{align*}
        \nabla^2 p = rhs \hspace{1.0in}
         & p(-L/2) = a \\
         & \ddx{}{x_i} p(L/2)= b \\
         & \text{periodic BCs in all other directions}
      \end{align*}
    \item 1dsine: 1D sine wave.
      \begin{align*}
        \nabla^2 p = - c_x^2 \sin \left(c_x x \right) \hspace{1.0in}
        & \text{periodic BCs in all directions}
      \end{align*}
    \item 3dsines: A product of 3 1D sine waves to create a 3D source. The
          solution for this is approximate due to the difficulty of matching
          the periodic length exactly.
      \begin{align*}
        \nabla^2 p = -\left( c_x^2 + c_y^2 + c_z^2 \right)
        \sin \left(c_x x \right) \sin \left(c_y y \right)
        \sin \left(c_z z \right)  \hspace{1.0in} \\
        \text{periodic BCs in all directions}
      \end{align*}
    \item cfdhw: Similar to linearpgrad but with one set of periodic BCs
                 replaced by a Neumann one. The initial conditions are 
                 different as well; noise is applied to the boundaries.
      \begin{align*}
        \nabla^2 p = 0 \hspace{1.0in}
         & p(-L_y/2) = a \\
         & p(L_y/2) = b \\
         & \ddx{}{z} p(-L_z/2)= \ddx{}{z} p(L_z/2) = 0 \\
         & \text{periodic BCs in x}
      \end{align*}
  \end{enumerate}
  \item testdir1D: Assign a direction for one of the 1D cases. Options are
  \begin{enumerate}
    \item x1: variation in the x1 direction, x2 and x3 periodic.
    \item x2: variation in the x2 direction, x1 and x3 periodic.
    \item x3: variation in the x3 direction, x1 and x2 periodic.
  \end{enumerate}
  \item srcflucs: If yes, add small fluctuations to the source term.
  \item phiflucs: If yes, add small fluctuations to the initial guess for phi.
\end{description}

\subsection{Grid generation program}
Running the grid generation program is done with the combination of gridgen.x
and gridgen.ini. All settings for the grid generation are given in gridgen.ini.
An example file is given below:
\input{autogen/gridgenini}

\textbf{Domain block options}
\begin{description}
  \item NX: number of grid points in the X1 direction. Ghost points are added
            automatically so this number should be rich in powers of 2
            (256 instead of 258). 
  \item NY: number of grid points in the X2 direction.
  \item NZ: number of grid points in the X3 direction.
\end{description}

\textbf{Grid block options}
\begin{description}
  \item Lx: Domain length in the X1 direction.
  \item Ly: Domain length in the X2 direction.
  \item Lz: Domain length in the X3 direction.
  \item x0: Set the grid origin in X1. Options are
  \begin{enumerate}
    \item start: x0 = 0
    \item end: x0 = -xL
    \item center: x0 = -xL/2
  \end{enumerate}
  \item y0: Set the grid origin in X2.
  \item z0: Set the grid origin in X3.
  \item XGridtype: Type of grid in the X1 direction. Options are
  \begin{enumerate}
    \item Uniform: Uniform spacing, $\triangle x = \frac{Lx}{NX+1}$
    \item stretch1: Hyperbolic sine stretching: $x(i) = -x0 + \frac{xL}{NX+1}*
                   \sinh \left( 1.75 \frac{i-1}{2*(NX+1)} \right)$
  \end{enumerate}
  \item YGridtype: Type of grid in the X2 direction.
  \item ZGridtype: Type of grid in the X3 direction.
\end{description}

\subsection{Interpolation Program}
coming soon... this needs an ini file.


\section{Statistics}

\section{References}
Add a set of references to thesis chapters describing the code and a list of
papers written using the code.

\appendix
\section{Description of all subroutines organized by directory} \label{app:allsubs}

%\subsection{Modules}
%\input{autogen/modulecontents}

\subsection{DNS Solver Subroutines}
\input{autogen/dnssubs}

\subsection{Pressure Solver Subroutines}
\input{autogen/psolvesubs}

\subsection{Pressure Solver Test Program Subroutines}
\input{autogen/ptestsubs}

\subsection{MPI Decomposition Directory}
\input{autogen/decompsubs}

\subsection{Commonly Used Subroutines: utils directory}
\input{autogen/utilssubs}

\subsection{Grid Generation Program}
\input{autogen/gridprogram}

\subsection{Interpolation Program}
\input{autogen/interpprogram}

\section{Post-Processing Tools}
\subsection{Matlab files for the turbulent wake, post/wake/wakematlabfiles}
\input{autogen/matlabwakepost}

\section{Acknowledgements}
The original code was written by Bernard Bunner at the U. of Michigan. It was 
adapted by Kyle Brucker at UCSD for use with free shear turbulent flows. The
modified program was made to match the code developed by Sankar Basak. Kyle
continued to update the code, releasing a major revision in June 2009. Matt 
de Stadler helped test, validate and verify, organize, and document the code. 
Matt will take over as owner of the code when Kyle graduates.

%\newpage

%\bibliography{biblio}
%\bibliographystyle{plain}

\end{document}
